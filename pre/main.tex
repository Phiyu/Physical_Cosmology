\documentclass[aspectratio=169, 10pt]{beamer}

% --- Theme & Colors ---
\usetheme{Madrid}
\usecolortheme{dolphin} 
\usefonttheme{professionalfonts}

% --- Packages ---
\usepackage{amsmath}
\usepackage{amssymb}
\usepackage{graphicx}
\usepackage{hyperref}
\usepackage{booktabs}
\usepackage{tikz}

\usepackage[backend=bibtex,style=authoryear]{biblatex}
\addbibresource{references.bib}


% --- Metadata ---
\title[EFT of LSS]{The Effective Field Theory of Large Scale Structure}
% \subtitle{Precision Cosmology Beyond the Linear Regime}
\author[Hongfei Yu]{Hongfei Yu}
\institute[USTC]{
    Department of Astrophysics,\\
    School of Physical Science,\\
    University of Science and Technology of China
}
\date{\today}

% --- Custom Commands ---
\newcommand{\kvec}{\vec{k}}
\newcommand{\xvec}{\vec{x}}
\newcommand{\deld}{\delta_D}

\begin{document}

% -----------------------------------------------------------------------------
% Title Slide
% -----------------------------------------------------------------------------
\begin{frame}
    \titlepage
\end{frame}
\begin{frame}
    \frametitle{Outline}
    \tableofcontents
\end{frame}

% -----------------------------------------------------------------------------
% Introduction & Motivation
% -----------------------------------------------------------------------------
\section{Motivation}

\begin{frame}{The Era of Precision Cosmology}
    \begin{columns}
        \column{0.5\textwidth}
        \begin{itemize}
            \item \textbf{Success of Linear Theory:} 
            On very large scales ($k \to 0$), standard linear perturbation theory describes the universe remarkably well (e.g., CMB).
            
            \item \textbf{The Frontier:} 
            Upcoming surveys (DESI, Euclid, LSST) will probe smaller scales with unprecedented precision.
            
            \item \textbf{The Challenge:}
            Linear theory breaks down at the non-linear scale:
            $$ k_{NL} \sim 0.1 - 0.3 \, h/\text{Mpc} $$
            Here, $\delta \sim \mathcal{O}(1)$, and loop corrections become divergent.
        \end{itemize}
        
        \column{0.5\textwidth}
        \begin{figure}
            \centering
            % Placeholder for P(k) comparison graph
            % Suggestion: Use Figure 1 from Philcox notes or Senatore's P_theory/P_NL plot
            % \framebox{\parbox{0.9\textwidth}{\centering \vspace{3cm} \textbf{[Insert Plot: P(k) Linear vs Non-linear]} \\ Shows breakdown at $k \sim 0.1$ \\ \vspace{3cm}}}
            \includegraphics[width = 1\textwidth]{fig1.png}
        \end{figure}
    \end{columns}
\end{frame}

\begin{frame}{Why not just use N-body Simulations?}
    Simulations are powerful, but they have limitations (Senatore):
    \vspace{0.5cm}
    \begin{itemize}
        \item \textbf{Computational Cost:} Running high-resolution simulations for every point in parameter space is prohibitive.
        \item \textbf{Systematic Errors:} Simulations have their own percent-level inaccuracies.
        \item \textbf{Baryonic Physics:} We cannot reliably simulate complex baryonic effects (star formation, feedback) from first principles on cosmological scales.
        \item \textbf{Understanding:} We want an analytical \textit{theory}, not just a numerical black box.
    \end{itemize}
\end{frame}

% -----------------------------------------------------------------------------
% Standard Perturbation Theory
% -----------------------------------------------------------------------------
\section{Standard Perturbation Theory (SPT)}

\begin{frame}{Standard Perturbation Theory (SPT)}
    SPT treats dark matter as a \textbf{perfect pressureless fluid} at all scales.
    
    \begin{block}{The Equations (Euler \& Continuity)}
        $$ \dot{\delta} + \nabla \cdot [(1+\delta)\vec{v}] = 0 $$
        $$ \dot{\vec{v}} + \mathcal{H}\vec{v} + (\vec{v}\cdot\nabla)\vec{v} = -\nabla \Phi $$
    \end{block}
    
    \begin{itemize}
        \item Expanded iteratively: $\delta = \delta^{(1)} + \delta^{(2)} + \delta^{(3)} + \dots$
        \item \textbf{The Problem:} Loop integrals integrate over \textit{all} momenta, including $k \to \infty$.
        \item This assumes the fluid is perfect even at the scale of galaxies and stars, which is physically incorrect (shell crossing, virialization).
    \end{itemize}
\end{frame}

% -----------------------------------------------------------------------------
% The EFT Philosophy
% -----------------------------------------------------------------------------
\section{The EFT Philosophy}

\begin{frame}{The Dielectric Analogy}
    \begin{columns}
        \column{0.5\textwidth}
        How do we describe light propagating through a material?
        \begin{itemize}
            \item \textbf{Microscopic View:} Trillions of atoms, electrons, complex QM interactions.
            \item \textbf{Macroscopic View:} We don't solve the Schrödinger equation for every atom!
            \item We use \textbf{Effective Parameters}: Polarizability $\to$ Dielectric constant $\epsilon$.
        \end{itemize}
        
        \column{0.5\textwidth}
        \begin{figure}
            \centering
            % Placeholder for Dielectric analogy image from Senatore slides
            % \framebox{\parbox{0.8\textwidth}{\centering \vspace{2cm} \textbf{[Insert Image: Dielectric Material]} \\ Atoms vs Smooth Medium \\ \vspace{2cm}}}
            \includegraphics[width = 0.8\textwidth]{fig2.png}
        \end{figure}
    \end{columns}
    \vspace{0.5cm}
    \textbf{The Lesson:} We can describe large-scale physics without knowing the details of small-scale non-linearities (galaxy formation), provided we introduce effective coefficients.
\end{frame}

\begin{frame}{Formalism: Smoothing the Universe (Baumann)}
    We split the fields into long-wavelength (IR) and short-wavelength (UV) modes using a smoothing scale $\Lambda$:
    $$ \delta_L(\xvec) = \int d^3y \, W_\Lambda(|\xvec-\vec{y}|) \delta(\vec{y}) $$
    
    Applying this to the Euler equation generates a new term due to non-linearity:
    $$ [\rho v_i v_j]_\Lambda \neq \rho_L v_{Li} v_{Lj} $$
    
    \begin{alertblock}{The Effective Stress Tensor}
        $$ \dot{v}_{Li} + \mathcal{H}v_{Li} + \dots = -\nabla \Phi_L \alert{- \frac{1}{\rho_L} \partial^j [\tau_{ij}]_\Lambda} $$
    \end{alertblock}
    $\tau_{ij}$ encodes the backreaction of small-scale physics (velocity dispersion, virialization) on large scales.
\end{frame}

% -----------------------------------------------------------------------------
% The Effective Stress Tensor
% -----------------------------------------------------------------------------
\section{Constructing the EFT}

\begin{frame}{Parameterizing the Unknown: The Stress Tensor}
    Since we cannot compute $\tau_{ij}$ from first principles (UV physics is non-linear), we expand it in terms of long-wavelength operators allowed by symmetries (Equivalence Principle):
    
    $$ \langle \tau_{ij} \rangle \sim \underbrace{p_b \delta_{ij}}_{\text{Background Pressure}} + \underbrace{c_s^2 \delta_L \delta_{ij}}_{\text{Sound Speed}} - \underbrace{\frac{c_{vis}^2}{H} \partial_i v_{Lj}}_{\text{Viscosity}} + \dots $$
    
    \begin{itemize}
        \item \textbf{$c_s^2$ (Effective Sound Speed):} Represents the resistance of small-scale structures to gravitational collapse.
        \item \textbf{Renormalization:} This term leads to a counterterm in the power spectrum:
        $$ P_{EFT}(k) = P_{linear} + P_{1-loop}^{SPT} \alert{- 2c_s^2 k^2 P_{linear}} $$
        \item $c_s^2$ is a \textbf{free parameter} measured from data or simulations.
    \end{itemize}
\end{frame}

% -----------------------------------------------------------------------------
% Results
% -----------------------------------------------------------------------------
\section{Results \& Success}

\begin{frame}{EFT vs. SPT vs. N-body Simulations}
    \begin{columns}
        \column{0.65\textwidth}
        \begin{figure}
            \centering
            % Placeholder for the key result plot (e.g., Senatore's plot showing 1% accuracy)
            % \framebox{\parbox{0.95\textwidth}{\centering \vspace{4cm} \textbf{[Insert Image: $P_{theory}/P_{NL}$]} \\ Comparing SPT, EFT 1-loop, EFT 2-loop \\ \vspace{4cm}}}
            \includegraphics[width = 1.0\textwidth]{fig3.png}
        \end{figure}
        
        \column{0.35\textwidth}
        \textbf{Key Findings:}
        \begin{itemize}
            \item SPT fails at $k \sim 0.1 \, h/\text{Mpc}$.
            \item \textbf{1-loop EFT} extends reach to $k \sim 0.3$.
            \item \textbf{2-loop EFT} reaches $k \sim 0.6$ with $\sim 1\%$ precision.
            \item We gain access to $\sim 100\times$ more modes than linear theory!
        \end{itemize}
    \end{columns}
\end{frame}

\begin{frame}{IR-Resummation: Handling Bulk Flows}
    \begin{itemize}
        \item \textbf{The Problem:} Large-scale bulk flows displace matter by $\sim 10 \, \text{Mpc}$. This "smears" the Baryon Acoustic Oscillations (BAO) peak.
        \item Standard Eulerian perturbation theory cannot handle these large displacements (convergence issues).
        \item \textbf{EFT Solution:} We can resum these infrared (IR) modes non-perturbatively.
        \item \textbf{Result:} The EFT correctly predicts the damping of the BAO oscillations, matching observations without ad-hoc smoothing.
    \end{itemize}
\end{frame}

% -----------------------------------------------------------------------------
% Conclusion
% -----------------------------------------------------------------------------
\section{Conclusion}

\begin{frame}{Summary \& Outlook}
    \begin{itemize}
        \item \textbf{A Rigorous Theory:} EFTofLSS is not a model; it is a rigorous perturbation theory based on the hierarchy of scales ($k \ll k_{NL}$).
        
        \item \textbf{UV Physics Encapsulated:} The complex formation of halos and galaxies (and baryonic physics) is encapsulated in effective coefficients (like $c_s^2$), which we fit to data.
        
        \item \textbf{High Precision:} It allows us to analytically compute the power spectrum to much higher $k$, unlocking vast amounts of cosmological information from future surveys.
        
        \item \textbf{Outlook:} 
        \begin{itemize}
            \item Application to Redshift Space Distortions (RSD).
            \item Biased tracers (Galaxies/Halos).
            \item Constraining Primordial Non-Gaussianity.
        \end{itemize}
    \end{itemize}
\end{frame}

\begin{frame}{Reading Material}
    \textbf{Core References for this Presentation:}
    
    \vspace{1em}
    
    The theoretical framework discussed today relies heavily on the foundational lectures by Leonardo Senatore, which introduce the effective fluid description derived from the Boltzmann equation \parencite{senatore_lectures}.
    
    \vspace{1em}
    
    For practical applications, including IR-resummation and data comparison, we follow the comprehensive introduction provided by Oliver Philcox \parencite{philcox_intro}.
    
    \vspace{2em}

    \printbibliography
\end{frame}

\begin{frame}[plain]
    \centering
    \Huge \textbf{Thank You!} \\
\end{frame}



\end{document}